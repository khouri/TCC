% Para digitar e compreender em portugues os comandos
\usepackage[brazil]{babel}
\usepackage[utf8]{inputenc}
\usepackage{indentfirst}
\usepackage[T1]{fontenc}

%Para incluir figuras
\usepackage{graphicx}

%Símbolos matemáticos.
\usepackage{amsmath}
\usepackage{amssymb}
\usepackage{amsthm}

%Permite customizar a legenda das figuras.
\usepackage[small,bf]{caption}

%Define as margens da folha, tamanho...
\usepackage[lmargin=2cm,tmargin=2cm,bmargin=2cm,rmargin=2cm]{geometry}


%Put abntex package


% To type URL with linebreak at special characters.
\usepackage{url}

% Improved and customizable hyphenation patterns.
%\usepackage{hyphenat}
%\hyphenation{pe-rio-do res-pon-sá-vel}

% Pacote de cores
\usepackage{color}
\usepackage{xcolor}        

%Permite a inclusão de código fonte de forma bonita.
\usepackage{listings}
\lstset{		
	language = C, 				% Linguagem de programação
	keywordstyle = \color{blue}, 		% Estilo das palavras chaves
	commentstyle = \color{dkgreen}, 	% Estilo dos Comentários
	stringstyle =  \color{orange}, 		% Estilo de Strings
	numbers=left,
        stepnumber=1,
        firstnumber=1,
        numberstyle=\tiny,
        extendedchars=true,
        breaklines=true,
        frame=tb,
        basicstyle=\footnotesize,
        stringstyle=\color{red},
        showstringspaces=false,
        otherkeywords={FILE, NULL, free, malloc, cos, sqrt, log}
        }
\renewcommand{\lstlistingname}{Código}
\renewcommand{\lstlistlistingname}{Lista de Listagens}


%Permite incluir uma lista do tipo Zebra

%\usepackage{listings}
%\usepackage{xcolor}


%Cria um comando de lista com duas cores de linhas em latex (Estilo 'Zebra').
%\newcommand\realnumberstyle[1]{}

%\makeatletter
%\newcommand{\zebra}[3]{%
 %   {\realnumberstyle{#3}}%
 %   \begingroup
 %   \lst@basicstyle
 %   \ifodd\value{lstnumber}%
 %       \color{#1}%
 %   \else
 %       \color{#2}%
 %   \fi
 %       \rlap{\hspace*{\lst@numbersep}%
 %       \color@block{\linewidth}{\ht\strutbox}{\dp\strutbox}%
 %       }%
 %   \endgroup
%}%
%\makeatother


%% Referências bibliográficas e afins
% Formatar as citações no texto e a lista de referências
\usepackage{natbib}

% Adicionar bibliografia, índice e conteúdo na Tabela de conteúdo
% Não inclui lista de tabelas e figuras no índice
\usepackage[nottoc,notlof,notlot]{tocbibind}


%inserir codigo fonte bonito.
\usepackage{minted}

%\usemintedstyle{borland}
\usemintedstyle{default}
\definecolor{bg_gray}{rgb}{0.95,0.95,0.95}

\renewcommand{\theFancyVerbLine}{
 \sffamily\textcolor[rgb]{0.5,0.5,0.5}{\scriptsize\arabic{FancyVerbLine}}}

