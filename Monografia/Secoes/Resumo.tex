% Mandatory (4.1.9 Resumo na língua vernácula).
\chapter*{Resumo}
%\begin{resumo}

No presente trabalho analisamos o comportamento assimptótico de uma equação diferencial estocástica cuja componente determinística apresenta bi-estabilidade.
O termo determinístico puro é integrável e sua solução é apresentada nesse trabalho. Escolhemos uma condição em que a distância entre as duas soluções assimptóticas estáveis são equidistantes à solução instável.

À equação diferencial ordinária original, introduzimos ruído nos moldes de Langevin, ou seja, pela adição de uma 
"força aleatória", transformando o sistema inicial numa equação diferencial estocástica.
A solução da equação diferencial estocástica é obtida numericamente pelo método de Euler-Maruyama, introduzimos o ruído pelo método de Langevin, adicionando uma força aleatória à equação diferencial ordinária biestável. 
A amplitude da perturbação \textit{($\sigma$)} é um parâmetro livre do nosso sistema e possui três condições distintas.
Pequenos valores de $\sigma$ garantem que a solução somente flutue em torno do limite assimptótico estável correspondente á condição inicial da equação. Por outro lado, para grandes valores de $\sigma$ , a solução flutua em torno do limite assimptótico instável. Por fim, em caso de valores intermediários da amplitude de flutuação, notamos que a solução, para grandes valores de tempo e independente da condição inicial, alterna flutuações aleatórias em torno de uma ou outra das soluções assimptóticas 
estáveis.

Nosso modelo pode ser aplicado para o estudo de sistemas biológicos associados ao desenvolvimento celular. Em geral, uma certa função celular é tratada como uma região de estabilidade. No entanto, sistemas celulares apresentam ruído.
Um exemplo de sistema que apresenta multi-estabilidade são as células de metazoários em que células se especializam em realizar funções específicas dentro do organismo. Podemos associar, a cada possível função celular, uma etiqueta que, em terminologia matemática, podemos chamar de ponto estável.

Se escrevermos uma EDO multi-estável cuja variável dependente é utilizada como etiqueta celular, cada um de seus limites assimptóticos representa uma de suas possíveis funções.
 
 
 
%
%equações diferenciais estocásticas. Solucionamos por meio analitico e numérico (algoritmo de Euler) uma 
%equação diferencial ordinária (que origina a equação diferencial estocástica).
%
%Utilizando o algoritmo de Euler obtivemos resultados da equação diferencial ordinária, 
%plotamos e analisamos os gráficos compreendendo o comportamento da equação em tempos infinitos.
%Após compreender o comportamento do termo determinístico da equação diferencial estocástica adicionamos
%um ruído na mesma obtendo dessa forma a equação diferencial estocástica.
%
%Para solucionarmos a equação diferencial estocástica aplicamos o algoritmo de Euler Maruyama, 
%dessa forma, conseguimos obter resultados sobre seu comportamento em tempos infinitos. 
%
%Por fim comparamos o comportamento das equações: ordinária e estocástica e constatamos que ocorre uma migração de regimes assimptóticos em determinados intervalos de tempo para amplitudes de ruído suficientemente grandes.

%\end{resumo}