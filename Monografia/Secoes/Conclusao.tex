\chapter{Conclusão}
Neste trabalho de conclusão de curso, apresentamos a solução exata e numérica de uma 
EDO, equação (\ref{ModeloDeterministicoEDO}), que apresenta bi-estabilidade. A solução exata foi obtida por integração direta da EDO, enquanto 
a solução numérica foi obtida pela implementação do algoritmo de Euler em linguagem C. 
O segundo ponto importante foi estudar o efeito de perturbações estocásticas sobre esse sistema. Introduzimos uma perturbação aleatória à EDO inicial, nos moldes de Langevin, e obtivemos uma EDE, equação (\ref{triangulo}). A equação estocástica obtida foi investigada numericamente, utilizando-se o algoritmo de Euler-Maruyama. 
Mostramos que o regime estacionário do sistema aleatório guarda dependência com a intensidade do ruído. 

No caso da EDO biestável, estudamos o seu comportamento de equilíbrio, e sua dependência das 
condições iniciais do sistema. Para condições iniciais positivas, ou seja $0\le x(0) \le +\infty$, 
a solução da equação apresenta um equilibrio estável em $x = a$, para condições iniciais negativas, isto é $ 0 \ge x(0) \ge -\infty$, o sistema apresenta equilíbrio estável em $x=-a$; o sistema ainda apresenta equilíbrio instável para a soluções onde a condição inicial nula.

Ao incluirmos uma perturbação aleatória, verificamos que o comportamento assimptótico (grandes valores de tempo) do
sistema representado pela EDE depende da amplitude de ruído adicionada ao sistema.
Valores pequenos do ruído geram flutuações em torno de um limite assimptótico correspondente à
condição inicial do sistema, enquanto grandes valores da amplitude de ruído causam flutuações
em torno do limite assimptótico de equilibrio instável. É surpreendente que, para valores
intermediários da amplitude de flutuação, o comportamento assimptótico resulte em flutuações alternadas em torno
de ambos os pontos de equilíbrio estável. Pretendemos aplicar estes resultados para o estudo do
processo de determininação do destino celular.